\documentclass{article}

\usepackage{graphicx}
\usepackage[hidelinks]{hyperref}
\usepackage[a4paper, total={6in, 8in}]{geometry}
\usepackage[slovak]{babel}
\usepackage{caption}
\usepackage{subcaption}

\graphicspath{./include/}

\renewcommand{\figurename}{Obr.}
\renewcommand{\contentsname}{Obsah}

\begin{document}

\begin{titlepage}
	\null\vfill

	\begin{center}
		{\Huge Regulácia výšky hladiny }
		\vskip 2cm

		{\Large Cvičenie č. 10}
		\vskip 0.5cm

		{\large Spojité procesy}
	\end{center}

	\vfill
	\vfill

	\begin{flushright}
		Filip Lobpreis \\
		Matúš Machata \\
		\small\today\\
	\end{flushright}
	\hfill
\end{titlepage}

\thispagestyle{empty}
\clearpage

\tableofcontents
\thispagestyle{empty}
\clearpage

\section{Zadanie}
\label{sec:zadanie}
\pagenumbering{arabic}

\begin{figure}[!htbp]
	\begin{center}
		\includegraphics[width=0.8\textwidth]{./include/zadaniep1.png}
	\end{center}
	\caption{Prvá časť zadania z~cvičenia č. 9 z~predmetu spojité procesy}
	\label{fig:zadanie1}
\end{figure}

\begin{figure}[!htbp]
	\begin{center}
		\includegraphics[width=0.8\textwidth]{./include/zadaniep2.png}
	\end{center}
	\caption{Druhá časť zadania z~cvičenia č. 9 z~predmetu spojité procesy}
		\label{fig:zadanie2}
\end{figure}

\clearpage

\section{Teória}
\label{sec:teoria}

V~tomto zadaní je našou úlohou experimentálne overiť návrh regulácie výšky hladiny na~laboratórnom
modeli hydraulického systému a~overiť navrhované riešenie. Pri~tomto zadaní sme použili už~preddefinovanú schému
v~programe \textit{Simulink} (Obr.~\ref{fig:schema}).

\begin{figure}[!htbp]
	\begin{center}
		\includegraphics[width=0.8\textwidth]{./include/schema.png}
	\end{center}
	\caption{Schéma modelu z~cvičenia č. 9 z~predmetu spojité procesy}
	\label{fig:schema}
\end{figure}

V~tomto zapojení vidíme viacero vstupných signálov, tie sú už~preddefinované. \textbf{href} referenčná respektíve
žiadaná hodnota výšky hladiny. \textbf{Ventil hore} reprezentuje hodnotu vstupného prietoku do nádrže, jeho hodnoty
sú zadávané v percentách. \textbf{Ventil dole}, tato hodnota reprezentuje prietok výstupného prietoku nádrže.
Rovnako ako vstupný prietok je zadávaný v percentách. Hodnoty, ktoré sa budú meniť v priebehu meraní sú parametre
\textit{PID} regulátora.

$$ G_R(s) = K_P \left( 1 + \frac{1}{T_is} + T_ds \right) $$

Proporcionálna zložka \textbf{P} regulátora je rovná zložke $K_P$, Integračná zložka \textbf{I} je počítaná vzťahom
$\frac{K_P}{T_i}$, posledná derivačná zložka \textbf{D} regulátora je rovná $K_P T_d [-]$.
Vyššie spomenutý vzťah môžeme teda prepísať do tvaru:

$$ G_R(s) = P  + \frac{1}{s}I + Ds $$

\clearpage

\subsection{Úvod do priebehu simulácie}
\label{subsec:priebehSimulacie}

Počas priebehu simulácie sa spomínaná žiadaná hodnota výšky hladiny mení. Priebeh zmien tejto veličiny je znázornený
na obrázku Obr.~\ref{fig:ziadanaHodnota}. Hodnota žiadanej hodnoty postupne skokovo rastie po hodnotu 70 cm. Prvá
žiadaná hodnota je nastavená na 10cm čas tejto časti simulácie je nastavený na 90 sekúnd. Druha žiadaná hodnota
je nastavená na 30cm a jej čas na ustálenie hodnoty je 100 sekúnd. Každá ďalšia žiadaná hodnota výšky hladiny
je rozdelená po 20cm segmentoch a doba počas ktorej sa má hodnota ustáliť je nastavená na 100 sekúnd. Po dosiahnutí
najvyššej žiadanej hodnoty sa začne nádrž pomaly vypúšťať. Prvá časť je vypustenie nádrže na výsku hladiny 40 cm.
Čas vypúšťania na túto hodnotu je nastavený taktiež na 100 sekúnd. Druha časť vypúšťania trvá taktiež 100 sekúnd
a hodnota výšky hladiny je nastavená na 20 cm. V poslednej časti sa má nádrž úplne vypustiť. Koniec simulácie nastáva
20 sekúnd po nastavení nulovej žiadanej hodnoty hladiny nádrže.

\begin{figure}[!htbp]
	\begin{center}
		\includegraphics[width=\textwidth]{./include/ziadana_hodnota.png}
		\caption{Priebeh referenčnej teploty v~priebehu 600 sekúnd.}
		\label{fig:ziadanaHodnota}
	\end{center}
	\hfill
\end{figure}

\newpage

\section{Merania}
\label{sec:merania}

\subsection{Meranie 1}
\label{sec:meranie1}

V~prvom zadaní sme zvolili prednastavené hodnoty parametrov regulátora. Tie sú zo zadania nastavené nasledovne:

\begin{center}
\begin{tabular}{ |c|c| }
 \hline
 $K_p [-]$ & 15 \\
 $T_i [-]$ & 20 \\
 $T_d [-]$ & 0,05 \\
 \hline
\end{tabular}
\end{center}

Výsledok simulácie môžeme vidieť na~obrázku Obr.~\ref{fig:m1}.

\begin{figure}[!htbp]
	\begin{center}
		\includegraphics[width=\textwidth]{./include/meranie1.png}
	\end{center}
	\caption{Graf žiadanej a~meranej hodnoty výšky hladiny na~snímači v~prvom meraní.}
	\label{fig:m1}
\end{figure}


Toto nastavenie regulátorov ovplyvnilo okolie, v ktorom sme nastavovali hodnoty v ďalších meraniach. Po zistení
priebehu výstupného signálu zobrazeného na Obr.~\ref{fig:m1} sme sa snažili dosiahnuť čo najrýchlejšiu reguláciu
výstupného signálu zo systému. Na dosiahnutie tohto cieľu sme zväčšili derivačnú zložku regulátora.
Tá zabezpečuje najrýchlejšiu odpoveď regulátora na zmenu výstupného signálu. Táto zložka zároveň zavádza šum
do akčného zásahu systému. Ďalšia zložka, ktorá zabezpečuje postupný rast akčného zásahu systému je integračná
zložka. Tá sa skladá ako bolo spomenutý v sekcii~\ref{sec:teoria} z parametrov $K_p$ a $T_i$. Pričom táto zložka
priamo úmerne závisí od zložky $K_p$ a nepriamo od $T_i$. Zmenu, ktorú sme vykonali bola zmena parametru
zosilnenia regulátora $K_p$.


\clearpage

\subsection{Meranie 2}
\label{sec:meranie2}

V druhom meraní sme si zvolili hodnoty:
Výsledok simulácie môžeme vidieť na obrázku Obr.~\ref{fig:m1}.

\begin{center}
\begin{tabular}{ |c|c| }
 \hline
 $K_p [-]$ & 30 \\
 $T_i [-]$ & 20 \\
 $T_d [-]$ & 0,1 \\
 \hline
\end{tabular}
\end{center}

\begin{figure}[!htbp]
	\begin{center}
		\includegraphics[width=\textwidth]{./include/meranie2.png}
	\end{center}
	\caption{Graf žiadanej a~meranej hodnoty výšky hladiny na~snímači v~druhom meraní.}
	\label{fig:m2}
\end{figure}

Pri druhom meraní sme si zmenili parametre regulátora. V porovnaní s prvým meraním si môžeme všimnúť,
že preregulovanie pri vypúšťaní nádrže nebolo také viditeľné. Zároveň nábeh výšky vodného stĺpca v nádrži
z 0cm na 10cm bol rýchlejší. Tieto výhody prišli aj s jednou nevýhodou. Zašumenie výstupného signálu
systému (výšky hladiny) bolo viditeľnejšie.

\clearpage

\subsection{Meranie 3}
\label{sec:meranie3}

V treťom meraní sme si zvolili hodnoty:

\begin{center}
\begin{tabular}{ |c|c| }
 \hline
 $K_p [-]$ & 50 \\
 $T_i [-]$ & 20 \\
 $T_d [-]$ & 0,5 \\
 \hline
\end{tabular}
\end{center}

Výsledok simulácie môžeme vidieť na obrázku Obr.\ref{fig:m3}.

\begin{figure}[!htbp]
	\begin{center}
		\includegraphics[width=\textwidth]{./include/meranie3.png}
	\end{center}
	\caption{Graf žiadanej a~meranej hodnoty výšky hladiny na~snímači v~treťom meraní.}
	\label{fig:m3}
\end{figure}

V treťom meraní sme sa snažili dosiahnuť podobný ale výraznejší efekt ako v meraní druhom. Prestavenie
parametrov regulátora však nami chcený efekt neprinieslo. Čo sa ale zlepšilo bolo skoršie ustálenie výšky
hladiny na žiadanej hodnote pri prvom skoku žiadanej hodnoty. Na druhej strane pri vypúšťaní nádrže sa jej
hladina pohybovala vo väčšom rozptyle okolia žiadanej hodnoty.

\clearpage

\subsection{Meranie 4}
\label{sec:meranie4}

V štvrtom meraní sme si zvolili hodnoty. 

\begin{center}
\begin{tabular}{ |c|c| }
 \hline
 $K_p [-]$ & 20 \\
 $T_i [-]$ & 10 \\
 $T_d [-]$ & 0,1 \\
 \hline
\end{tabular}
\end{center}

Výsledok simulácie môžeme vidieť na obrázku Obr.\ref{fig:m4}.

\begin{figure}[!htbp]
	\begin{center}
		\includegraphics[width=\textwidth]{./include/meranie4.png}
	\end{center}
	\caption{Graf žiadanej a~meranej hodnoty výšky hladiny na~snímači vo~štvrtom meraní.}
	\label{fig:m4}
\end{figure}

Vo štvrtom meraní sme zvolili odlišný postup. Parameter $K_p$ sme zmenili na hodnotu 20. Parameter $T_i$
sme zmenili na hodnotu 10. Tento postup zabezpečí zväčšenie integračnej zložky pričom proporcionálna
zložka a derivačná zložka ostanú nezmenené. Pri tejto konfigurácii je ustálenie signálu rovnako
bez viditeľného preregulovania pri vypúšťaní nádrže a ustálenie zvyšujúcej sa hladiny nádrže sa ustáli
pomerne rýchlo pri prvej skokovej zmene žiadanej hodnoty výšky hladiny.

\clearpage

\subsection{Meranie 5}
\label{sec:meranie5}

V piatom meraní sme si zvolili hodnoty. 

\begin{center}
\begin{tabular}{ |c|c| }
 \hline
 $K_p [-]$ & 50 \\
 $T_i [-]$ & 20 \\
 $T_d [-]$ & 0,1 \\
 \hline
\end{tabular}
\end{center}

Výsledok simulácie môžeme vidieť na obrázku Obr.\ref{fig:m5}.

\begin{figure}[!htbp]
	\begin{center}
		\includegraphics[width=\textwidth]{./include/meranie5.png}
	\end{center}
	\caption{Graf žiadanej a~meranej hodnoty výšky hladiny na~snímači v~piatom meraní.}
	\label{fig:m5}
\end{figure}

V piatom meraní sme skúsili zväčšiť hodnoty každého parametra jeden a pol násobne oproti minulému meraniu 
pričom integračnú zložku sme zmenšili ešte o polovicu. Tento postup zaviedol rýchly nábeh a z prechodovej
charakteristiky sa stal kmitavý proces s preregulovaním. Tento jav nebolo vidno voľným okom na prítoku nádrže.

\clearpage

\subsection{Meranie 6}
\label{sec:meranie6}

V šiestom meraní sme si zmenili hodnotu uzatvorenia výpustného ventilu zo 70\% na 50 \%, hodnoty
\textit{Kp}, \textit{Ti} a \textit{Td} sme ponechali rovnaké. Výsledok simulácie môžeme vidieť na Obr.~\ref{fig:m6}.

\begin{center}
\begin{tabular}{ |c|c| }
 \hline
 $K_p [-]$ & 50 \\
 $T_i [-]$ & 20 \\
 $T_d [-]$ & 0,1 \\
 \hline
 $Vstupny\_ventil[\%]$ & 100 \\
 $Vystupny\_ventil[\%]$ & 50 \\
 \hline
\end{tabular}
\end{center}

Výsledok simulácie môžeme vidieť na obrázku Obr.~\ref{fig:m6}.

\begin{figure}[!htbp]
	\begin{center}
		\includegraphics[width=\textwidth]{./include/meranie6.png}
	\end{center}
	\caption{Graf žiadanej a~meranej hodnoty výšky hladiny na~snímači v~šiestom meraní.}
	\label{fig:m6}
\end{figure}

Po diskusii s cvičiacim sme sa dohodli na zmene hodnoty výstupného ventilu a odsimulovania tejto zmeny.
V šiestom meraní sme si prestavili výstupný ventil z 70\% na 50\%. Tento postup zaviedol do procesu
obmedzenie, ktoré sa prejavilo na zrýchlení napúšťania nádrže a spomalenia vypúšťania vody z nádrže.

\clearpage

\subsection{Meranie 7}
\label{sec:meranie7}

V siedmom meraní sme si zmenili hodnotu uzatvorenia výpustného ventilu a hodnoty \textit{Kp}, \textit{Ti}
a \textit{Td} sme ponechali rovnaké.

\begin{center}
\begin{tabular}{ |c|c| }
 \hline
 $K_p [-]$ & 50 \\
 $T_i [-]$ & 20 \\
 $T_d [-]$ & 0,1 \\
 \hline
 $Vstupny\_ventil[\%]$ & 100 \\
 $Vystupny\_ventil[\%]$ & 80 \\
 \hline
\end{tabular}
\end{center}

Výsledok simulácie môžeme vidieť na Obr.~\ref{fig:m7}.

\begin{figure}[!htbp]
	\begin{center}
		\includegraphics[width=\textwidth]{./include/meranie7.png}
	\end{center}
	\caption{Graf žiadanej a~meranej hodnoty výšky hladiny na~snímači v~siedmom meraní.}
	\label{fig:m7}
\end{figure}

V posledom siedmom meraní sme zmenili hodnotu výstupného ventilu z 50\% na 80\%. Hodnoty parametrov
regulátora sme ponechali rovnaké ako posledných dvoch meraniach, aby sme vedeli urovnať efekt zmeny
výstupného filtra. Tato zmena spôsobila pomalšie napúšťanie nádrže a zároveň rýchlejšie vypustenie
vody z nádrže.

\clearpage

\section{Zhrnutie}
\label{sec:zhrnutie}


Môžeme si všimnúť, že pri základných hodnotách prítoku a odtoku zmena PID regulátora pomocou zmeny $K_p$, $T_i$ a $T_d$ má minimálny vplyv na rýchlosť napúštania a vypúštania nádrže. Môžeme tým hlavne ovplyvňovať stabilitu vodnej hladniny na požadovanej hodnote. 

Pri takto navrhnutom systéme môžme PID regulátorom ovplyvniť stromsť napúštania a vypúštania iba do určitej miery lebo je limitovaná zvoleným ventilom. Ako môžeme vidieť v meraní v sekcii \ref{sec:meranie6} znížením rýchlosti odtoku sa nám síce podarilo zlepšiť rýchlosť napúštania nádrže ale od času 400s systém nestíhal vypúštať vodu na požadovanú hodnotu. Naopak v poslednom meraní v sekcii \ref{sec:meranie7} môžeme vidieť, že zvýšenie odtoku spôsobilo to, že síce vieme rýchlejšie vypúštať vodu ale systém mal probem s napúštaním, hlavne v čase od 300s po 400s kedy je hladina vody najvyššia.

% \tab V tomto zadaní sme sledovali ako vplíva PID regulátor, prítok a odtok vodny v nádrži. 
% Prvé meranie sme uskutočnili s prednastavenými hodnotami ktoré môžeme vidieť v tabuľke \ref{sec:meranie1}, 
% ako môžme vidieť na Obr.\ref{fig:m1} pri vypúštaní je nedostatočné regulovanie a výska hladiny výrazne klesne pod požadovanú hodnotu.  

% Toto sme sa pokúsili eliminovať v ďaľšom meraní \ref{sec:meranie2} zvýšením hodnoty $K_p$ na 30 a $T_d$ na 0.1, čo nám celkom dobre pomohlo a ako môžme vidieť na Obr. \ref{fig:m2} hladina vody už neklesá tak výrazne pod požadovanú hodnotu. 

% Keďže toto malo pozitívny vplyv na systém v daľšom meraní sme zase tieto hodnoty trochu zvýšili.


\end{document}

